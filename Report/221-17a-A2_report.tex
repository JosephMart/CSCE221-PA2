%% LyX 2.2.2 created this file.  For more info, see http://www.lyx.org/.
%% Do not edit unless you really know what you are doing.
\documentclass[english]{article}
\usepackage{mathptmx}
\usepackage[T1]{fontenc}
\usepackage[latin9]{inputenc}
\usepackage{geometry}
\geometry{verbose,tmargin=0.8in,bmargin=0.8in,lmargin=1in,rmargin=1in,headheight=0in,headsep=0in}
\usepackage{babel}
\usepackage{array}
\PassOptionsToPackage{normalem}{ulem}
\usepackage{ulem}
\usepackage[unicode=true]
 {hyperref}

\makeatletter

%%%%%%%%%%%%%%%%%%%%%%%%%%%%%% LyX specific LaTeX commands.
%% Because html converters don't know tabularnewline
\providecommand{\tabularnewline}{\\}

%%%%%%%%%%%%%%%%%%%%%%%%%%%%%% Textclass specific LaTeX commands.
\newenvironment{lyxcode}
{\par\begin{list}{}{
\setlength{\rightmargin}{\leftmargin}
\setlength{\listparindent}{0pt}% needed for AMS classes
\raggedright
\setlength{\itemsep}{0pt}
\setlength{\parsep}{0pt}
\normalfont\ttfamily}%
 \item[]}
{\end{list}}

\makeatother

\begin{document}
\begin{center}
{\Large{}Project Cover Page }
\par\end{center}{\Large \par}

This project is a group project with up to three students per team.
For each group member, please print first and last name and e-mail
address. 

{\large{}1. ~~~~~ ~~~~~~~~~~~~~~~~~~~~~~~~~~~~~~~~~~~~~~~}\\
{\large \par}

{\large{}2.~~~~~~~~~~~~~~~~~~~~~~~~~~~~~~~~~~~~~~~~~~~~~~}\\
{\large \par}

{\large{}3.~~~~~~~}{\Large{}~~~~~~~~~~~~~~~~~~~~~~~~~~~~~~~~~~}{\Large \par}

\vspace{0.2in}
Please write how each member of the group participated in the project.

{\large{}1. ~~~~~ ~~~~~~~~~~~~~~~~~~~~~~~~~~~~~~~~~~~~~~~}\vspace{0.2in}
{\large{}}\\
\vspace{0.2in}

{\large{}2.~~~~~~~~~~~~~~~~~~~~~~~~~~~~~~~~~~~~~~~~~~~~~~}\vspace{0.2in}
{\large{}}\\
{\large \par}

{\large{}3.~~~~~~~}{\Large{}~~~~~~~~~~~~~~~~~~~~~~~~~~~~~~~~~~}\vspace{0.2in}

Please list all sources: web pages, people, books or any printed material,
which you used to prepare a report and implementation of algorithms
for the project. \bigskip{}

\begin{center}
\begin{tabular}{|c|p{4in}|}
\hline 
Type of sources: & \tabularnewline
\hline 
\hline 
 & \tabularnewline
People & \tabularnewline
 & \tabularnewline
\hline 
 & \tabularnewline
Web Material (give URL) & \tabularnewline
 & \tabularnewline
\hline 
 & \tabularnewline
Printed Material  & \tabularnewline
 & \tabularnewline
\hline 
 & \tabularnewline
Other Sources  & \tabularnewline
 & \tabularnewline
\hline 
\end{tabular}
\par\end{center}

\vspace{0.2in}
I certify that I have listed all the sources that I used to develop
solutions to the submitted project report and code.

Your signature \hfill{}Typed Name \hfill{}Date\hfill{}~~~~~~~

\vfill{}
I certify that I have listed all the sources that I used to develop
a solution to the submitted project and code.

Your signature \hfill{}Typed Name\hfill{}Date\hfill{}~~~~~~~~~

\vfill{}
I certify that I have listed all the sources that I used to develop
solution to the submitted project and code.

Your signature \hfill{}Typed Name \hfill{}Date\hfill{}~~~~~~~~

\newpage{}
\begin{center}
\textbf{\large{}CSCE 221 Programming Assignment}{\large{} }\textbf{\large{}2
(200 points)}
\par\end{center}{\large \par}

\begin{center}
\textit{Program and Reports due February 19th by 11:59pm}
\par\end{center}
\begin{itemize}
\item \textbf{Objective}

In this assignment, you will implement five sorting algorithms: selection
sort, insertion sort, bubble sort, shell sort and radix sort in C++.
You will test your code using varied input cases, time the implementation
of sorts, record number of comparisons performed in the sorts, and
compare these computational results with the running time of implemented
algorithms using Big-O asymptotic notation. You will be required to
manage this project with GitHub.
\item \textbf{General Guidelines}
\begin{enumerate}
\item This project can be done in groups of at most three students. Please
use the cover sheet at the previous page for your hardcopy report.
\item The supplementary program is packed in \texttt{221-A2-code.tar} which
can be downloaded from the course website. You need to ``untar''
the file using the following command on linux:
\begin{lyxcode}
tar~xfv~221-A2-code.tar~~
\end{lyxcode}
It will create the directory \texttt{221-A2-code}. You will be required
to push the files in this directory to your team's repository. For
more information on this, see the \href{https://tamucomputerscience.github.io/CSCE221-SupplementaryMaterial/source/git/gittutorial.html}{Git tutorial}.
\item Make sure that your code can be compiled using a C++ compiler running
on a linux machine before submission because your programs will be
tested on such a linux machine. Use \texttt{Makefile} provided in
the directory to compile C++ files by typing the following command:
\begin{lyxcode}
make
\end{lyxcode}
You may clean your directory with this command:
\begin{lyxcode}
make~clean
\end{lyxcode}
\item When you run your program on a linux machine, use Ctrl+C (press Ctrl
with C) to stop the program. Do NOT use Ctrl+Z, as it just suspends
the program and does not kill it. We do not want to see the department
server down because of this assignment.
\item Supplementary reading
\begin{enumerate}
\item Lecture note: \href{file:http://courses.cs.tamu.edu/teresa/csce221/pdf-lectures/algorithm-analysis_rev.pdf}{Introduction to Analysis of Algorithms}
\item Lecture note: \href{http://courses.cs.tamu.edu/teresa/csce221/pdf-lectures-print/linear_sorts-4.pdf}{Sorting in Linear Time}
\item \href{https://tamucomputerscience.github.io/CSCE221-SupplementaryMaterial/source/git/gittutorial.html}{Git tutorial}
\end{enumerate}
\item Submission guidelines
\begin{enumerate}
\item Electronic copy of all your code, 15 types of input integer sequences,
and reports in LyX and PDF format should be in the directory \texttt{221-A2-code}.
This command typed in the directory \texttt{221-A2-code} will create
the tar file (\texttt{221-A2-code-submit.tar}) for the submission
to CSNet:
\begin{lyxcode}
make~tar~
\end{lyxcode}
\item Your program will be tested on TA's input files.
\end{enumerate}
\end{enumerate}
\item \textbf{Code }
\begin{enumerate}
\item In this assignment, the sort program reads a sequence of integers
either from the screen (standard input) or from a file, and outputs
the sorted sequence to the screen (standard output) or to a file.
The program can be configured to show total running time and/or total
number of comparisons done in the sort. 
\item This program does not have a menu but takes arguments from the command
line. The code for interface is completed in the template programs,
so you only have to know how to execute the program using the command
line.\pagebreak{}

The program usage is as follows. \emph{Note that options do not need
to be specified in a fixed order.}

\textbf{Usage:} 
\begin{lyxcode}
./sort~{[}-a~ALGORITHM{]}~{[}-f~INPUTFILE{]}~{[}-o~OUTPUTFILE{]}~{[}-h{]}~{[}-d{]}~{[}-p{]}~{[}-t{]}~{[}-c{]}~
\end{lyxcode}
\textbf{Example:} 
\begin{lyxcode}
./sort~-h

./sort~-a~S~-f~input.txt~-o~output.txt~-d~-t~-c~-p

./sort~-a~I~-t~-c

./sort
\end{lyxcode}
\textbf{Options:} 
\begin{lyxcode}
-a~ALGORITHM:~Use~ALGORITHM~to~sort.~
\begin{lyxcode}
ALGORITHM~is~a~single~character~representing~an~algorithm:~

S~for~selection~sort

B~for~bubble~sort~

I~for~insertion~sort~

H~for~shell~sort~

R~for~radix~sort
\end{lyxcode}
-f~INPUTFILE:~Obtain~integers~from~INPUTFILE~instead~of~STDIN~

-o~OUTPUTFILE:~Place~output~data~into~OUTPUTFILE~instead~of~STDOUT~

-h:~Display~this~help~and~exit~

-d:~Display~input:~unsorted~integer~sequence

-p:~Display~output:~sorted~integer~sequence

-t:~Display~running~time~of~the~chosen~algorithm~in~milliseconds~

-c:~Display~number~of~comparisons~(excluding~radix~sort)
\end{lyxcode}
\item \textbf{Format of the input data.} The first line of the input contains
a number \emph{n} which is the number of integers to sort. Subsequent
\emph{n} numbers are written one per line which are the numbers to
sort. Here is an example of input data:

\texttt{5 // this is the number of lines below = number of integers
to sort}~\\
\texttt{7}~\\
\texttt{-8}~\\
\texttt{4}~\\
\texttt{0}~\\
\texttt{-2}
\item \textbf{Format of the output data.} The sorted integers are printed
one per line in increasing order. Here is the output corresponding
to the above input:

\texttt{-8}~\\
\texttt{-2}~\\
\texttt{0}~\\
\texttt{4}~\\
\texttt{7}
\item (\emph{50 points}) Your tasks include implementing the following five
sorting algorithms in corresponding cpp files.
\begin{enumerate}
\item selection sort in \texttt{selection-sort.cpp}
\item insertion sort in \texttt{insertion-sort.cpp}
\item bubble sort in \texttt{bubble-sort.cpp}
\item shell sort in \texttt{shell-sort.cpp}
\item radix sort in \texttt{radix-sort.cpp}

\begin{enumerate}
\item Implement the radix sort algorithm that can sort 0 to $(2^{16}-1)$
but takes input $-2^{15}$ to $(2^{15}-1)$ . 
\item About radix sort of negative numbers: ``You can shift input to all
positive numbers by adding a number which makes the smallest negative
number zero. Apply radix sort and next make a reverse shift to get
the initial input.''
\end{enumerate}
\end{enumerate}
\item (\emph{20 points}) Generate the sets of the sizes $10^{2}$, $10^{3}$,
$10^{4}$, and $10^{5}$ integers in three different orders.
\begin{enumerate}
\item random order
\item increasing order
\item decreasing order
\end{enumerate}
HINT: The standard library \texttt{<cstdlib>} provides functions \texttt{srand()}
and \texttt{rand()} to generate random numbers.
\item Measure the average number of comparisons (excluding radix sort) and
average running times of each algorithms on the $12$ integer sequences. 
\begin{enumerate}
\item (\emph{20 points}) Insert additional code into each sort (excluding
radix sort) to count the number of \textbf{\emph{comparisons performed
on input integers}}. The following tips should help you with determining
how many comparisons are performed.
\begin{enumerate}
\item You will measure 3 times for each algorithm on each sequence and take
average
\item Insert the code that increases number of comparison \texttt{num\_cmps++
}typically in an \texttt{if} or a loop statement
\item Remember that C++ uses the shortcut rule for evaluating boolean expressions.
A way to count comparisons accurately is to use comma expressions.
For instance
\end{enumerate}
\begin{lyxcode}
while~(i~<~n~\&\&~(num\_cmps++,~a{[}i{]}~<~b))
\end{lyxcode}
HINT: If you modify \texttt{sort.cpp} and run several sorting algorithms
subsequently, you have to call resetNumCmps() to reset number of comparisons
between every two calls to \texttt{s->sort()}.
\item Modify the code in\texttt{ sort.cpp} so that it repeatedly measures
the running time of \texttt{s->sort()}.
\begin{enumerate}
\item You will measure roughly $10^{7}$ times for each algorithm on each
sequence and take the average. You have to run for the same number
of rounds for each algorithm on each sequence, and make sure that
each result is not 0.
\item When you measure the running time of sorting algorithms, please reuse
the input array but fill with different numbers. Do not allocate a
new array every time, that will dramatically slower the program.
\item To time a certain part of the program, you may use functions \texttt{clock()}
defined in header file \texttt{<ctime>}, or \texttt{gettimeofday()}
defined in \texttt{<sys/time.h>}. Here are the examples of how to
use these functions. The timing part is also completed in the template
programs. However, you will apply these function to future assignments.
\end{enumerate}
\end{enumerate}
The example using \texttt{clock()} in \texttt{<ctime>}:
\begin{lyxcode}
\#include~<ctime>
\begin{lyxcode}
...

clock\_t~t1,~t2;

t1~=~clock();~//~\emph{start~timing}

...

/{*}~\emph{operations~you~want~to~measure~the~running~time}~{*}/

...

t2~=~clock();~//~\emph{end~of~timing}

double~diff~=~(double)(t2~-~t1)/CLOCKS\_PER\_SEC;

cout~<\textcompwordmark{}<~\textquotedbl{}The~timing~is~\textquotedbl{}~<\textcompwordmark{}<~diff~<\textcompwordmark{}<~``~ms''~<\textcompwordmark{}<~endl;
\end{lyxcode}
\end{lyxcode}
The example using \texttt{gettimeofday()} in \texttt{<sys/time.h>}:
\begin{lyxcode}
\#include~<sys/time.h>
\begin{lyxcode}
...

struct~timeval~start,~end;

...

gettimeofday(\&start,0);~//~\emph{start~timing}

...

/{*}~\emph{operations~you~want~to~measure~the~running~time}{*}/

...

gettimeofday(\&end,0);~//~\emph{end~of~timing}

double~diff~=~(end.tv\_sec~-~start.tv\_sec)~

~~~~~~~~~~~~~+~(double)(end.tv\_usec~-~start.tv\_usec)/1e6;

cout~<\textcompwordmark{}<~\textquotedbl{}The~timing~is~\textquotedbl{}~<\textcompwordmark{}<~diff~<\textcompwordmark{}<~``~sec''~<\textcompwordmark{}<~endl;
\end{lyxcode}
\end{lyxcode}
\end{enumerate}
\item \textbf{Report (110 points)}

Write a report that includes all following elements in your report.
\begin{enumerate}
\item (5 points) A brief description of assignment purpose, assignment description,
how to run your programs, what to input and output.
\item (5 points) Explanation of splitting the program into classes and \emph{a
description of C++ object oriented features or generic programming
used in this assignment}.
\item (5 points)\textbf{ Algorithms.} Briefly describe the features of each
of the five sorting algorithms.
\item (20 points) \textbf{Theoretical Analysis.} Theoretically analyze the
time complexity of the sorting algorithms with input integers in decreasing,
random and increasing orders and fill the second table. Fill in the
first table with\textbf{ }the time complexity\textbf{ }of the sorting
algorithms when inputting the best case, average case and worst case.
Some of the input orders are exactly the best case, average case and
worst case of the sorting algorithms. State what input orders correspond
to which cases. You should use big-O asymptotic notation when writing
the time complexity (running time).

\begin{tabular}{|c|c|c|c|}
\hline 
\textbf{Complexity} & ~~best~~ & average & worst\tabularnewline
\hline 
\hline 
Selection Sort &  &  & \tabularnewline
\hline 
Insertion Sort &  &  & \tabularnewline
\hline 
Bubble Sort &  &  & \tabularnewline
\hline 
Shell Sort &  &  & \tabularnewline
\hline 
Radix Sort &  &  & \tabularnewline
\hline 
\end{tabular}~~~~~~~~~~~~~%
\begin{tabular}{|c|c|c|c|}
\hline 
\textbf{Complexity} & ~~inc~~ & ~~ran~~ & ~~dec~~\tabularnewline
\hline 
\hline 
Selection Sort &  &  & \tabularnewline
\hline 
Insertion Sort &  &  & \tabularnewline
\hline 
Bubble Sort &  &  & \tabularnewline
\hline 
Shell Sort &  &  & \tabularnewline
\hline 
Radix Sort  &  &  & \tabularnewline
\hline 
\end{tabular}

\uline{inc:} increasing order; \uline{dec:} decreasing order;
\uline{ran:} random order
\item (65 points)\textbf{ Experiments.} 
\begin{enumerate}
\item Briefly describe the experiments. Present the experimental running
times (\textbf{RT}) and number of comparisons (\textbf{\#COMP}) performed
on input data using the following tables. \medskip{}

\begin{tabular}{|c||l|c|c||c|c|c||c|c|c|}
\hline 
\textbf{RT} & \multicolumn{3}{l||}{Selection Sort} & \multicolumn{3}{c||}{Insertion Sort} & \multicolumn{3}{c|}{Bubble Sort}\tabularnewline
\hline 
$n$ & inc & ran & dec & inc & ran & dec & inc & ran & dec\tabularnewline
\hline 
\hline 
$100$ &  &  &  &  &  &  &  &  & \tabularnewline
\hline 
$10^{3}$ &  &  &  &  &  &  &  &  & \tabularnewline
\hline 
$10^{4}$ &  &  &  &  &  &  &  &  & \tabularnewline
\hline 
$10^{5}$ &  &  &  &  &  &  &  &  & \tabularnewline
\hline 
\end{tabular}\medskip{}

\begin{tabular}{|c||c|c|c||c|c|c|}
\hline 
\textbf{RT} & \multicolumn{3}{c||}{Shell Sort} & \multicolumn{3}{c|}{Radix Sort}\tabularnewline
\hline 
$n$ & inc & ran & dec & inc & ran & dec\tabularnewline
\hline 
\hline 
$100$ &  &  &  &  &  & \tabularnewline
\hline 
$10^{3}$ &  &  &  &  &  & \tabularnewline
\hline 
$10^{4}$ &  &  &  &  &  & \tabularnewline
\hline 
$10^{5}$ &  &  &  &  &  & \tabularnewline
\hline 
\end{tabular}

\medskip{}

\uline{}%
\begin{tabular}{|c||c|c|c||c|c|c|}
\hline 
\textbf{\#COMP} & \multicolumn{3}{c||}{Selection Sort} & \multicolumn{3}{c|}{Insertion Sort}\tabularnewline
\hline 
\hline 
$n$ & inc & ran & dec & inc & ran & dec\tabularnewline
\hline 
$100$ &  &  &  &  &  & \tabularnewline
\hline 
$10^{3}$ &  &  &  &  &  & \tabularnewline
\hline 
$10^{4}$ &  &  &  &  &  & \tabularnewline
\hline 
$10^{5}$ &  &  &  &  &  & \tabularnewline
\hline 
\end{tabular}

\medskip{}

\uline{}%
\begin{tabular}{|c||c|c|c||c|c|c|}
\hline 
\textbf{\#COMP} & \multicolumn{3}{c||}{Bubble Sort} & \multicolumn{3}{c|}{Shell Sort}\tabularnewline
\hline 
\hline 
$n$ & inc & ran & dec & inc & ran & dec\tabularnewline
\hline 
$100$ &  &  &  &  &  & \multicolumn{1}{c|}{}\tabularnewline
\hline 
$10^{3}$ &  &  &  &  &  & \tabularnewline
\hline 
$10^{4}$ &  &  &  &  &  & \tabularnewline
\hline 
$10^{5}$ &  &  &  &  &  & \tabularnewline
\hline 
\end{tabular}

\uline{inc:} increasing order; \uline{dec:} decreasing order;
\uline{ran:} random order
\item For each of the five sort algorithms, graph the running times over
the three input cases (inc, ran, dec) versus the input sizes ($n$);
and for each of the first four algorithms graph the numbers of comparisons
versus the input sizes, totaling in 9 graphs. 

HINT: To get a better view of the plots, \emph{use logarithmic scales}
for both x and y axes.
\item To compare performance of the sorting algorithms you need to have
another 3 graphs to plot the results of all sorts for the running
times for \emph{each} of the input cases (inc, ran, dec) separately.

HINT: To get a better view of the plots, \emph{use logarithmic scales}
for both x and y axes.
\end{enumerate}
\item (5 points) \textbf{Discussion.} Comment on how the experimental results
relate to the theoretical analysis and explain any discrepancies you
note. Is your computational results match the theoretical analysis
you learned from class or textbook? Justify your answer. Also compare
radix sort's running time with the running time of four comparison-based
algorithms.
\item (5 points) \textbf{Conclusions.} Give your observations and conclusion.
For instance, which sorting algorithm seems to perform better on which
case? Do the experimental results agree with the theoretical analysis
you learned from class or textbook? What factors can affect your experimental
results?
\end{enumerate}
\item \textbf{(20 points) Use of Git }
\begin{enumerate}
\item (10 points) Host the final working version of your project on your
team's GitHub repository. \textbf{This repository must be private.} 
\item (10 points) Demonstrate knowledge of branching. Create branches for
at least three of the six sorting algorithms. Once the sorting algorithm
is working, merge the branch with the master branch but DO NOT delete
the branch used to implement the algorithm.
\end{enumerate}
Each member of your team should have roughly equal amount of contributions.
This would be approximately implementing two of the six sorting algorithms
and roughly equal contributions to the final report. This information
will be used to assign the final grade for this assignment. 

If you are unfamiliar with git or need a refresher, see the \href{https://tamucomputerscience.github.io/CSCE221-SupplementaryMaterial/source/git/gittutorial.html}{Git tutorial}. 
\item \texttt{\textbf{\large{}Quick Start Guide}}{\large \par}
\begin{enumerate}
\item Choose a team member from each group (max of three members) to create
a\textbf{ private} repository on \texttt{github.tamu.edu}.
\item This same team member will add each team member as a collaborator
to the repository. 
\item One of the team members should download the supplementary code from
the website. \texttt{Untar} the supplementary code from the website. 
\item Inside the directory created by untaring the file, use the command
'\texttt{git init}' to create a local repository in this directory. 
\item Use the command '\texttt{git remote add origin} {[}address to your
repository{]}' to create the link between your local repository and
the remote repository.
\item Use the command '\texttt{git push origin master}' to push the supplementary
code to your team's repository.
\item Your team can begin to create branches for the sorting algorithms. 
\item See \href{https://tamucomputerscience.github.io/CSCE221-SupplementaryMaterial/source/git/gittutorial.html}{Git tutorial}
for more information.
\end{enumerate}
\end{itemize}

\end{document}
